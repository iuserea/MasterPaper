%!TEX root = ../csuthesis_main.tex
\keywordsen{Churn Prediction; User Preference; User Intervention; Deep Learning; Reinforcement Learning}
\categoryen{TP391}
%\itemcountcn{There are \totalfigures\ figures, \totaltables\ tables, and \total{citnum}\ citations in this thesis.}
%\addcontentsline{toc}{section}{ABSTRACT}
\begin{abstracten}\setlength{\baselineskip}{20pt}
	
	Internet card (IC) is a new business model proposed recently by the cooperation of cellular operators and Internet companies. Particularly, IC users can enjoy significant discounts or exemptions on target traffic budget, that generated by the applications of partner Internet companies.However, in China, because of the market of IC is becoming increasingly competitive, the demand of IC is gradually saturated, etc., the churn problem of IC user is getting worse which led to much economic loss. In order to automatically and efficiently retain users throughout the entire process, this paper conducts research on the churn prediction and the churn reason inference of Internet card users and the retention issues. Nevertheless, there are three challenges to addressing these issues:\par
	1)The Internet card users of communication operators are relatively large, distributed all over the country, with obvious heterogeneous characteristics, rich data types and fast growth. Therefore, how to predict potential churner of IC accurately and comprehensively has become a significant challenge.\par
	2)There are hundreds of reasons for Internet card users to churn, and only a few users have recorded the reasons for churning. How to reasonably and efficiently represent the user’s churn preference with a small amount of labeled data has become a thorny challenge.\par
	3)Due to the lack of user intervention interaction records, and the budget given by companies such as communication operators in the actual intervention process is often limited. Therefore, how to match appropriate retention measures to pre-churners can not only solve the problems in the user's use process, but at the same time not make the enterprise pay too much cost is a great challenge.\par
	To address these issues, we proposed a \underline{U}ser \underline{P}reference-Aware \underline{R}etention \underline{F}ramework. Specifically, the \emph{UPRF} framework includes the following three parts:
	\par
	1) Internet card churn prediction module based on self-attention mechanism.
	We firstly makes a detailed data analysis on churn behavior, churn reason, etc. Then, the IC user's portrait features and sequence features are extracted from the user data respectively. Finally, an Internet card user churn prediction model that combines principal component analysis algorithm and self-attention mechanism is designed and implemented.	
%	to output the list of potential churn users and the corresponding churn risk value list of the current month to the operator at the beginning of the month.
	\par
	2) Churn preference generation module based on ranking-weighted normalization technology. We firstly extracts 5 churn reasons that can be reflected by the data. Secondly, the preference characteristics that are more related to these churn reasons are selected. Then, a dual ranking discretization algorithm based on equal frequency binning was designed and implemented to present raw churn preference of pre-churners. Finally, the user-adaptive weight normalization algorithm is used to output the churn preference of the pre-churners.
	\par
	3) User preference-aware retention strategy matching module in a resource-limited contexts.	The retention strategy matching problem is firstly modeled as a multi-armed bandit problem. Then the strategy matching module receives the churn risk value given by the churn prediction module and the churn preference given by the churn preference generation module. Nextly, it initializes the reward generation model using a distribution based on domain knowledge. Finally, a preference-aware retention strategy matching algorithm in a resource-limited context is trained.
	\par
	Last but not least, extensive data-driven experiments demonstrate the efficacy of \emph{UPRF} in predicting potential IC churner and raising number of reserved user and total revenue of mobile commnications operator. Comparing to the best benchmark models, it imporved by 12\%, 25\% and 23\%, respectively. Furthermore, comprehensive robustness testing shows that \emph{UPRF} has good perfermance under different parameter settings.	
	
%Internet card (IC) is a new business model proposed recently by the cooperation of cellular operators and Internet companies. Particularly, IC users can enjoy significant discounts or exemptions on target traffic budget, that generated by the applications of partner Internet companies. However, in China, the market of IC is becoming increasingly competitive, the demand of IC is gradually saturated, the churn problem of IC user is getting worse which led to much economic loss. To address these issues, we proposed a \underline{U}ser \underline{P}reference-Aware \underline{R}etention \underline{F}ramework, named UPRF. It utilizes a data-driven approach, ranking-weighted normalization technology and reinforcement learning to predict potential churner, represent churn preference, match retention strategy, respectively. Specifically, the UPRF framework includes the following three parts:
%\par
%1) Internet card churn prediction module integrating principal component analysis algorithm and self-attention mechanism.
%In view of the challenges of large-scale user data and significant heterogeneous characteristics of the Internet card of the operator, this paper makes a detailed data analysis on churn behavior, churn reason, etc. Then, the IC user's static portrait features and sequence features are extracted from the user data respectively. Finally, an Internet card user churn prediction model that combines principal component analysis algorithm and self-attention mechanism is designed and implemented to output the list of potential churn users and the corresponding churn risk value list of the current month to the operator at the beginning of the month.
%\par
%2) Churn preference generation module based on ranking-weighted normalization technology. In view of the challenges of massive churn reasons of IC users and of IC users often churning due to multiple reasons, the paper first extracts 5 churn reasons that can be reflected by the data. Secondly, the preference characteristics that are more related to these churn reasons are selected. Then, a dual ranking discretization algorithm based on equal frequency binning was designed and implemented to present raw churn preference of pre-churners. Then the inactive pre-churners are excluded through the untrustworthy user filtering mechanism. Finally, the user-adaptive weight normalization algorithm is used to output the churn preference of the pre-churners to the operator and the retention strategy matching module.
%\par
%3) User preference-aware retention strategy matching module in a resource-limited contexts.
%Due to the limited budget of the marketing department of operators, problems merged such as the inability to retain pre-churners through low retention costs. In this paper, the retention strategy matching problem is firstly modeled as a multi-armed bandit problem. Then the strategy matching module receives the churn risk value given by the churn prediction module and the churn preference given by the churn preference generation module. Nextly, it initializes the reward generation model using a distribution based on domain knowledge. Finally, a preference-aware retention strategy matching algorithm in a resource-limited context is trained.
%\par
%Last but not least, extensive data-driven experiments demonstrate the efficacy of UPRF in predicting potential IC churner and raising number of reserved user and total revenue of mobile commnications operator. Comparing to the best benchmark models, it imporved by 12\%, 25\% and 23\%, respectively. Furthermore, comprehensive robustness testing shows that UPRF has good perfermance under different parameter settings.

	
%Internet card is a new business model recently proposed by mobile communication operators and internet companies in cooperation. It allows users to enjoy significant discounts or exemptions on the target traffic budget generated by cooperating internet companies. However, due to the increasingly fierce competition in the Chinese internet card market, demand is gradually saturating, and the problem of user churn is becoming more serious, resulting in significant economic losses. Therefore, this article focuses on predicting and intervening to retain internet card users.
%
%Firstly, the problem of user churn prediction and intervention strategy matching are formally modeled as binary classification and sequence decision problems, respectively, with maximizing the number of successfully retained users as the optimization objective. Then, considering the challenges of large-scale heterogeneous features of internet card user data, a significant framework for efficient, accurate, and comprehensive internet card user intervention is proposed, which is crucial for maintaining internet card users.
%
%To this end, this article designs a \underline{U}ser \underline{P}reference-Aware \underline{R}etention \underline{F}ramework (UPRF) that predicts internet card user churn in a data-driven manner, characterizes churn preferences based on weighted normalization techniques, and combines reinforcement learning algorithms to match intervention strategies for potential churn users. Specifically, the UPRF framework includes the following three parts:
%
%1) Design and implementation of an internet card user churn prediction module that integrates principal component analysis algorithm and self-attention mechanism.
% Considering the challenges of large-scale heterogeneous features of internet card user data, this article conducts detailed data analysis on churn behavior, extracts static portrait features and sequence features from user data, and finally designs and implements an internet card user churn prediction model that integrates principal component analysis algorithm and self-attention mechanism. At the beginning of each month, the model outputs a list of potential churn users and corresponding churn risk values to the operator.\par
%2) Design and implementation of a churn preference generation module based on rank-weighted normalization techniques. 
%Considering the challenges of multiple reasons for internet card user churn and the tendency for churn behavior to be a result of a combination of factors, this article first extracts five churn reasons that can be reflected in the data. Then, the relevant preference features are selected and a ranking normalization algorithm based on equi-frequency binning is designed and implemented. The raw churn preference vector of predicted churn users is obtained, and inactive predicted churn users are filtered out using an untrustworthy user filtering mechanism. Finally, the churn preference representation of predicted churn users is output to the operator and intervention strategy matching module using a user adaptive weighting normalization algorithm.\par
%3) Design and implementation of the user preference-aware intervention strategy matching module with limited resources context.
% In response to the challenges of crude retention measures designed by the marketing departments of telecom operators at present, which cannot retain potential churn users at low retention costs, this article first models the intervention strategy matching problem as a sequential decision problem. Then, it receives the churn risk values given by the Internet card user churn prediction module and the churn preference given by the Internet card pre-churn user churn preference generation module. It uses a domain-knowledge-based distribution initialization reward generation model and finally trains the preference-aware intervention strategy matching algorithm based on limited resource context.\par
%Finally, a large amount of data-driven experiments show that the UPRF framework can effectively increase the number of retained users and the total revenue of the telecom operator. In addition, a large number of robustness tests also show that the UPRF framework has good robustness under different parameter settings.

\end{abstracten}
