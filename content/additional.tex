%!TEX root = ../csuthesis_main.tex

{~}
\vspace{18pt}
\section{攻读学位期间主要的研究成果} % 无章节编号

\ifblindreview
% \noindent
% (盲审隐去作者相关具体信息)
\fi
\vspace{11pt}
\subsection*{一、学术论文}

\ifblindreview

% \noindent
% 第一作者:JCR 1 区 x 篇,会议 x 篇 \\{}
% 第二作者:JCR 1 区 x 篇,3 区 x 篇,4 区 x 篇,EI x 篇 

% \noindent
% 投稿状态: 
% IEEE Transactions on Image Processing 1篇(under review)\\{}
% IEEE Transactions on Circuits and Systems for Video Technology 1 篇(Accept with Minor Revision)

% 学位办老师要求用如下这种几乎算是单盲的格式,我也木有办法……
\subsubsection*{已录用/检索论文}
%\noindent
%第一作者:
%\begin{enumerate}[label={[\arabic*]},itemindent=2em,wide]
%	\item CSU Latex Template[J]. CSU player: 1(1):1-10. {\bfseries \heiti(SCI 录用,JCR 1 区)}
%	\item CSU Latex Template[J]. CSU player: 1(1):1-10. {\bfseries \heiti(SCI 检索,JCR 2 区)}
%\end{enumerate}
%第二作者:
%\begin{enumerate}[label={[\arabic*]},itemindent=2em,wide]
%	\item CSU Latex Template[J]. CSU player: 1(1):1-10. {\bfseries \heiti(SCI 检索,JCR 1 区)}
%	\item CSU Latex Template[J]. CSU player: 1(1):1-10. {\bfseries \heiti(SCI 检索,JCR 2 区)}
%\end{enumerate}
第五作者:
\begin{enumerate}[label={[\arabic*]},itemindent=2em,wide]
	\item Characterizing Internet Card User Portraits for Efficient Churn Prediction Model Design[J]. IEEE Transaction on Mobile Computing, 2023. {\bfseries \heiti(SCI 检索,JCR 1 区)}
	%	\item Director, \textbf{Daxia Mou}, Someone, Someother. XXXXXX[J]. Transactions on Image Processing. {\bfseries \heiti(SCI Under Review,JCR 1 区)}
	%	\item Director, \textbf{Daxia Mou}, Someone, Someother. XXXXXX[J]. Transactions on Circuits and Systems for Video Technology. {\bfseries \heiti(SCI Under Review,JCR 1 区)}
\end{enumerate}
%\noindent
%\subsubsection*{投稿状态论文}
%%\noindent
%第一作者:
%\begin{enumerate}[label={[\arabic*]},itemindent=2em,wide]
%	\item CSU Latex Template. XXX Transactions on CSU player. {\bfseries \heiti(SCI Under Review,JCR 1 区)}
%\end{enumerate}
%第二作者:
%\begin{enumerate}[label={[\arabic*]},itemindent=2em,wide]
%	\item CSU Latex Template. XXX Transactions on CSU player. {\bfseries \heiti(SCI Under Review,JCR 1 区)}
%	\item CSU Latex Template. XXX Transactions on CSU player. {\bfseries \heiti(SCI Under Review,JCR 2 区)}
%\end{enumerate}


\else
% 标准版本
\begin{enumerate}[label={[\arabic*]},itemindent=2em,wide]
	\item Fan Wu, Feng Lyu, Ju Ren, Peng Yang, \textbf{Kai Qian}, Shijie Gao, Yaoxue Zhang. Characterizing Internet Card User Portraits for Efficient Churn Prediction Model Design[J]. IEEE Transaction on Mobile Computing: 1(1):1-10. {\bfseries \heiti(SCI 检索,JCR1区)}
%	\item Director, \textbf{Daxia Mou}, Someone, Someother. XXXXXX[J]. Transactions on Image Processing. {\bfseries \heiti(SCI Under Review,JCR 1 区)}
%	\item Director, \textbf{Daxia Mou}, Someone, Someother. XXXXXX[J]. Transactions on Circuits and Systems for Video Technology. {\bfseries \heiti(SCI Under Review,JCR 1 区)}
\end{enumerate}
\fi

\vspace{22pt}
\subsection*{二、发明专利}
\ifblindreview
发明专利3项,其中2项已授权,1项已受理
\begin{enumerate}[label={[\arabic*]},itemindent=2em,wide]
	\item 导师第一发明人,本人第二发明人. 基于深度学习的用户流失预测方法及系统. 2022-11-01. 已授权.
	\item 导师第一发明人,本人第二发明人. 基于用户画像的互联网卡用户流失预测方法及系统. 2021-11-29. 已受理.
	\item 第五发明人. 一种数据驱动的互联网卡用户价值分类方法设备及介质. 2023-03-14. 已授权.
\end{enumerate}
\else
发明专利3项,其中2项已授权,1项已公开
\begin{enumerate}[label={[\arabic*]},itemindent=2em,wide]
	\item 吕丰,\textbf{钱凯},吴帆,任炬,张尧学. 基于深度学习的用户流失预测方法及系统. 申请号:CN2021112951915,授权公告号:CN 114022202 B
	\item 吕丰,\textbf{钱凯},吴帆,任炬,张尧学. 基于用户画像的互联网卡用户流失预测方法及系统. 申请号:2021112981395,公开号:CN 113962160 A
	\item 高世杰,张永敏,王姗姗,周杰钰,\textbf{钱凯}. 一种数据驱动的互联网卡用户价值分类方法设备及介质:ZL 202211513076.5[P]. 授权公告号:CN 115563555 B
\end{enumerate}
\fi

\ifblindreview
\else

\vspace{22pt}
%\subsection*{三、主持和参与的科研项目}
%\begin{enumerate}[label={[\arabic*]},itemindent=2em,wide]
%	\item 国家自然科学基金面上项目《XXXXXXXXXXXX》, 项目编号:XXXXXXXX,参与.
%\end{enumerate}

% {~}
\vspace{22pt}
\subsection*{三、个人获奖情况}
%\noindent
\begin{enumerate}[label={[\arabic*]},itemindent=2em,wide]
	\item 2020年度中南大学研究生一等奖学金
	\item 2021年度中南大学研究生二等奖学金
	\item 2022年度中南大学研究生二等奖学金
\end{enumerate}
\fi

%\newpage

\ifblindreview
\else

{~}
\vspace{18pt}

\newpage

\section{致谢} % 无章节编号
	白驹过隙,日月如梭,不知不觉间已经在透明计算实验室度过了三年难忘的研究生生活了。其中酸甜苦辣,百般滋味,只有自己知晓,不足为外人道也。但是,我今天能成功地取得硕士学位,离不开家人和以下老师,朋友的一贯支持和有力帮助。\par
	首先,我想感谢我的恩师,吕丰教授。在透明计算实验室的三年多的研究生生活中,吕老师不遗余力地指导我进行科研课题的研究,教授我解决问题的通用范式,身体力行地为我树立想要从事科研应当学习的榜样。在每次的组会讨论,外出谈判中,我都能从吕老师身上学习到了更加准确的思维逻辑和为人处世。\par
	其次,我想感谢实验室的吴帆、刘彤、周杰钰师兄和卢华丽、段思婧师姐,他们作为博士师兄、师姐,首先给了在科研和生活中非常多的指导以及帮助,并且他们身上刻苦钻研的科研精神和求实务真的科研态度深深地激励了我。\par
	然后,我想感谢实验室的同门,包括肖飞、高世杰、王艺锋、何骁豪、刘佳璇、张静,还有赵张梦茹、王姗姗、孟陈莹、唐程等师弟师妹们,他们每个人身上都有各自的闪光点和才华,和他们的朝夕相处中,我不仅感受到了蓬勃的青春,而且被处处的真情所感动,大家都十分团结,互帮互助,在一起时充满了欢声笑语。只可惜时不我待,我不得不和他们挥手告别。但是往昔的情谊都留在心中,不必多说。\par
	最后,我想感谢我的家人们,包括我的爸爸、妈妈、爷爷、奶奶、外公、外婆等,在我外出求学的第5到7个年头,包括之后的8到12个年头,无论我提出什么要求,只要是我自己愿意的,他们都无条件地支持我,并且及时地给予我物质上的支持和精神的援助。他们对我的爱,我也都深深铭刻在心中,而我也将我的实际行动回报他们对我的爱。\par
	最后的最后,感谢所有花时间阅读我硕士学位的老师、同学和朋友们,你们的意见和建议将使我变得更好。希望在下一个四年,你们能看到我更加优秀的博士学位论文。言尽于此,我继续去科研了。


% 重新设置正文行间距,因为前置部分设置时候行间距被改过
\renewcommand*{\baselinestretch}{1.0} % 几倍行间距
\setlength{\baselineskip}{20pt} % 基准行间距

%作者对给予指导、各类资助和协完成研究工以及提供种论文有 作者对给予指导、各类资助和协完成研究工以及提供种论文有 利条件的单位及个人表示感谢。
%
%致谢应实事求是,切忌浮夸与庸俗之词 。

\newpage
\fi
