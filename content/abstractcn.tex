%!TEX root = ../csuthesis_main.tex
% 设置中文摘要
\keywordscn{离网预测;用户偏好;用户干预;深度学习;强化学习}
\categorycn{TP391}
\itemcountcn{图 \totalfigures\ 幅,表 \totaltables\ 个,参考文献 \total{citnum}\ 篇}

\begin{abstractcn}\setlength{\baselineskip}{20pt}%\renewcommand{\baselinestretch}{1.0}
	
	互联网卡是最近通信运营商和互联网公司合作提出的一种新的商业模式,可以享受由合作互联网公司应用产生流量费用的大幅折扣或减免。然而,因为中国互联网卡市场竞争日趋激烈,需求逐步饱和等多种原因,用户离网/流失问题也日益严重,造成了较大的经济损失。为了全流程自动化高效地挽留住用户,本文针对互联网卡用户离网预测、离网原因推断和用户挽留问题展开研究。但是,要解决这些问题存在以下三个挑战:
	\par
	1) 通信运营商的互联网卡用户规模较大,分布全国各地,异构特征显著,数据类型丰富,速度增长快,因此如何精准且全面地预测出其中潜在的离网用户成为了一个十分重要的挑战。\par
	2) 互联网卡用户的离网原因高达上百种且只有少部分的离网用户有记录离网原因,如何在少量标注数据的情况下,如何合理、高效地表征用户离网偏好成为了一个棘手的挑战。\par
	3) 由于缺乏用户干预交互记录,并且通信运营商等企业在实际干预过程中给予的预算往往是有限的。因此如何向预离网用户匹配合适的挽留措施,既能解决用户使用过程中的痛点,同时又不会让企业方付出过高的成本是一个极大的挑战。\par
	为解决上述挑战,本文设计一种用户偏好感知的挽留框架(\emph{UPRF},\underline{U}ser \underline{P}reference-Aware \underline{R}etention \underline{F}ramework),具体而言,\emph{UPRF}框架包含以下三部分内容:\par
	1)基于自注意力机制的互联网卡用户离网预测模块。首先对离网行为等做了详细的数据分析,接着从用户数据中分别提取了用户画像特征和序列特征,最后设计和实现了一种融合主成分分析算法和自注意力机制的互联网卡用户离网预测模型。	\par
	2)基于排名加权归一化技术的离网偏好表征模块。首先抽取了5类能够被数据反映的离网原因,其次挑选了与这些离网原因相关的偏好特征,接着设计并实现了基于等频分箱的双重排名离散化算法,得到原始离网偏好向量,
	%然后通过不可信用户过滤机制排除了不活跃的预离网用户,
	最后使用用户自适应权重归一化算法表征预离网用户的离网偏好。		         \par
	3)资源有限上下文的用户偏好感知挽留策略匹配模块。首先将挽留策略匹配问题建模成多臂老虎机问题,然后接收由上述模块给出的离网风险和离网偏好,使用基于领域知识的分布初始化奖励生成模型,最后训练基于资源有限上下文的偏好感知的挽留策略匹配算法。	
	\par
	最后,大量数据驱动的实验表明\emph{UPRF}框架可以精准全面地预测潜在离网用户,有效提高挽留用户数和运营商收入总和。与最好的基准模型对比,其分别提升了12\%,25\%和23\%。此外,大量的健壮性测试也表明\emph{UPRF}框架在不同参数设置的情况下都具有较好的性能表现。	
%	由此,本文针对互联网卡用户离网预测和干预挽留问题展开研究。首先,离网预测问题和挽留策略匹配问题分别被形式化建模为二分类和多臂老虎机问题,并且以最大化挽留成功离网用户数为优化目标。
%然后考虑到运营商互联网卡用户数据规模较大,异构特征显著等挑战,需要一种高效,精准和全面的互联网卡用户干预框架,这对于互联网卡用户维系而言至关重要。	
%	互联网卡是最近通信运营商和互联网公司合作提出的一种新的商业模式,可以享受由合作互联网公司应用产生流量费用的大幅折扣或减免。然而因为中国互联网卡市场竞争日趋激烈,需求逐步饱和,用户离网问题也日益严重,造成了较大的经济损失。
%	为此,本文设计一种用户偏好感知的挽留框架(UPRF,\underline{U}ser \underline{P}reference-Aware \underline{R}etention \underline{F}ramework),通过数据驱动的方式对互联网卡用户进行离网预测,然后基于排名归一化技术表征离网偏好,最后结合强化学习算法对预离网用户进行挽留策略匹配。具体而言,UPRF框架包含以下三部分内容:
%		\par
%%\begin{enumerate}[label=\arabic*)]	
%%	\item 
%	1)融合主成分分析算法和自注意力机制的互联网卡用户离网预测模块。%设计与实现
%	针对运营商互联网卡的用户数据规模较大,异构特征显著等挑战,本文对离网行为等做了详细的数据分析,
%	接着从用户数据中分别提取了用户静态画像特征和序列特征,
%	最后设计和实现了一种融合主成分分析算法和自注意力机制的互联网卡用户离网预测模型,
%	以在月初向运营商输出当月的潜在离网用户名单和对应的离网风险值列表。
%	\par
%%	\item 
%	2)基于排名加权归一化技术的离网偏好生成模块。
%	针对互联网卡用户离网原因众多和往往因为多个原因的叠加而发生离网行为等挑战,
%	本文首先抽取了5个能够被数据反映的离网原因,
%	其次挑选了与这些离网原因较为相关的偏好特征,
%	接着设计并实现了基于等频分箱的双重排名离散化算法,
%	得到预离网用户的原始离网偏好向量,
%	然后通过不可信用户过滤机制排除了不活跃的预离网用户,
%	最后使用用户自适应权重归一化算法向运营商和挽留策略匹配模块输出预离网用户的离网偏好。		
%	\par
%%	\item 
%	3)资源有限上下文的用户偏好感知挽留策略匹配模块。
%	针对由于现阶段运营商营销部门预算有限,
%	无法通过较低的挽留成本将预离网用户留存下来等问题,
%	本文首先对挽留策略匹配问题建模成多臂老虎机问题,
%	然后接收由互联网卡用户离网预测模块给出的离网风险值和互联网卡预离网用户离网偏好生成模块给出的离网偏好,
%	使用基于领域知识的分布初始化奖励生成模型,
%	最后训练基于资源有限上下文的偏好感知的挽留策略匹配算法。		
%	\par
%%\end{enumerate}			
%	最后,大量数据驱动的实验表明UPRF框架可以精准全面地预测潜在离网用户,有效提高挽留用户数和运营商收入总和。与最好的基准模型对比,其分别提升了12\%,25\%和23\%。此外,大量的健壮性测试也表明UPRF框架在不同参数设置的情况下都具有较好的性能表现。
%LaTeX利用设置好的模板,可以编译为格式统一的pdf。目前国内大多出版社与高校仍在使用word,word由于其强大的功能与灵活性,在新手面对形式固定的论文时,排版、编号、参考文献等简单事务反而会带来很多困难与麻烦,对于一些需要通篇修改的问题,要想达到LaTeX的效率,对word使用者来说需要具有较高的技能水平。
%
%为了能把主要精力放在论文撰写上,许多国际期刊和高校都支持LaTeX的撰写与提交,新手不需要关心格式问题,只需要按部就班的使用少数符号标签,即可得到符合要求的文档。且在需要全篇格式修改时,更换或修改模板文件,即可直接重新编译为新的样式文档,这对于word新手使用word的感受来说是不可思议的。
%
%本项目的目的是为了创建一个符合中南大学研究生学位论文(博士)撰写规范的TeX模板,解决学位论文撰写时格式调整的痛点。
\end{abstractcn}