%!TEX root = ../csuthesis_main.tex
% 论文正文是主体,主体部分应从另页右页开始,每一章应另起页。一般由序号标题、文字叙述、图、表格和公式等五个部分构成。
\section{绪论}
\subsection{研究背景与意义}
\subsection{国内外研究现状}
\subsection{主要研究内容}
\subsection{论文组织结构}
\begin{figure}[hbt]
	\centering
	\includegraphics[width=1\textwidth]{Paper-Architecture_v1_1.pdf}
	\caption{论文组织架构图}
	\label{Fig:Paper_Architecture}
\end{figure}

\section{相关理论概述}
%\subsection{电信用户生命周期理论}%拉新、维护、干预
\subsection{用户画像}
\subsection{深度学习算法}
\subsubsection{卷积神经网络}
\subsubsection{循环神经网络}
\subsubsection{基于注意力机制的神经网络}
\subsection{强化学习算法}
\subsubsection{基于随机过程的多臂老虎机}
\subsubsection{基于上下文的多臂老虎机}
\subsection{本章小结}

\section{系统描述与问题建模}
\subsection{系统描述}
\subsubsection{离网用户预测模型}
\subsubsection{预离网用户偏好生成模型}
\subsubsection{预离网用户干预模型}
\subsection{问题建模}
\subsection{问题挑战}
\subsection{本章小结}

\section{用户数据分析和特征工程}
\subsection{平台描述}
\begin{figure}[hbt]
	\centering
	\includegraphics[width=1\textwidth]{Platform.pdf}
	\caption{大数据平台系统架构}
	\label{Fig:Data_Platform}
\end{figure}
在本章中, 我们会首先介绍平台架构,然后描述数据格式、规模等信息,接着进行了三个方面的数据分析,最后进行了相应的特征工程。
运营商们每天都会生产和存储巨量的数据,其中分为业务支持系统(BSS)和运营支持系统(OSS),这两者也构建了大数据平台的底层,从而用来提升业务和运营表现。具体来说,图\ref{Fig:Data_Platform}展示了流量运营商的大数据平台架构,其中包括数据层、业务逻辑层、核心算法层和应用层。在数据层中, 

\subsection{数据描述}
\begin{figure}[hbt]
	\centering
	\includegraphics[width=1\textwidth]{Ms-Data-Description-ICCP_v1_1.pdf}
	\caption{用户侧数据描述}
	\label{Fig:User-Side-Data}
\end{figure}
首先, 我们来描述一下用户侧数据, 如图\ref{Fig:User-Side-Data}所示。\par
\textbf{时间范围.} 我们拥有从2020年4~6月,11月~12月和2021年1~2月的7个月的数据。\par
\textbf{用户类型.} 我们过滤掉了政企用户、家庭用户和其他用户,只留下互联网卡个人用户。\par
\textbf{数据规模.} 在这7个月的数据中,一共有400万的互联网卡个人用户,400万条以月为粒度的属性表记录,35亿条以次为粒度的CDR(通话细节记录)表记录,400亿条以次为粒度的流量表记录,4亿条以月为粒度的APP(应用程度)表记录, 1000万条以次为粒度的停机表记录。
其中属性表记录的为用户属性数据,而其他四个表记录的为用户行为数据,尤其是流量表和CDR表的数据尤为珍贵,能够刻画用户的序列行为。但是从另一方面来说,如此大量的数据也给数据分析和模型训练推理带来了极大的硬件资源、方法性能、时间压力。\par
\textbf{具体字段.}\par
\textbf{数据用途.} \par
\textbf{存储类型.} 这些数据主要是以HDFS\par

\begin{figure}[hbt]
	\centering
	\includegraphics[width=1\textwidth]{Ms-Data-Description-ICSM_v2_1.pdf}
	\caption{物品侧数据描述}
	\label{Fig:Item-Side-Data}
\end{figure}
接着,我们来描述物品侧



\subsection{数据分析}
\subsubsection{用户静态属性分析}
\subsubsection{用户时序行为分析}
\subsubsection{用户异常行为分析}
\subsection{特征工程}
\subsubsection{静态特征工程}
\subsubsection{时序特征工程}
\subsection{本章小结}

\section{基于自注意力机制的互联网卡用户离网预测模型设计}
\subsection{系统描述与问题建模}
\subsection{基于PCA算法的特征降维算法}
\subsection{基于自注意力机制的嵌入向量表示}
\subsection{基于多层感知机的分类器设计}
\subsection{本章小结}

\section{预离网用户偏好生成算法设计}
\subsection{离网原因与偏好的相关性分析}
\subsection{离网偏好排名归一化}
\subsection{不可信用户过滤机制}
\subsection{本章小结}

\section{基于LinUCB的预离网用户干预模型设计}
\subsection{系统描述与问题建模}

\subsection{预离网用户偏好生成算法设计}
\subsubsection{离网原因与偏好的相关性分析}
\subsubsection{离网偏好排名归一化}
\subsubsection{不可信用户过滤机制}

\subsection{针对干预措施的特征工程}
\subsection{基于LinUCB的用户-干预措施匹配算法设计}
\subsubsection{动作空间}
\subsubsection{奖励机制设计}
\subsection{基于模拟干预结果机制的预训练}
\subsection{本章小结}

\section{实验评估与结果分析}
\subsection{实验设置}
\subsubsection{对比方案}
\subsubsection{评估指标}
\subsection{用户离网预测模型性能评估}
\subsection{预离网用户干预模型性能评估}
\subsection{参数影响}
\subsection{消融实验}
\subsection{本章小结}

\section{总结与展望}
\subsection{工作总结}
\subsection{未来工作展望}

\section{绪论}
\subsection{研究背景与意义}

目的是创建一个符合中南大学研究生学位论文(博士)撰写规范的LaTeX模板,解决学位论文撰写时格式调整的痛点。

已有珠玉在前,我们之所以还要重新造轮子,主要依据2022年4月18号学校下发的[《中南大学研究生学位论文撰写规范》中大研字【2022】8号] (http://oa.its.csu.edu.cn/Home/ReleaseMainText/9CFE8926B13143009D5EB424333AAD6C),重新修改了页面布局、字体类型和大小、标题内容,以期做到与 Word 模板尽可能的相似。\textbf{学校要求}:2022年起,申请博士、硕士学位的学位论文必须按新文件执行。主要修改如下:
\begin{itemize}
\item 按要求修订段落与各级标题间距;
\item 按要求修订中英文段落间距,章节间距,附录标题段落间距,研究成果及致谢标题间距,参考文献间距等;
\item 增加博士和硕士论文模板选项,只需要info.tex选择即可,方便使用;
\item 按新版撰写规范修改主要格式如下:修订目录章节标题间距;修订中英文段落间距;修订图片与表格标题的段落间距;
\item 按要求更新“学位论文版权使用授权书”;
\item 依据2022最新撰写规范修订封面和扉页-“封面”及“扉页”关于学科专业的表头更新为:一级学科/专业学位类别,二级学科/专业领域;
\item 依据专家意见修订定理和证明等环境,如”定理”使用小四黑体,编号随章节变化重新编号(如定理4-1),定理内容使用小四宋体,且内容行距与正文一致;”证明”无需编号,且以黑色小方块结尾;
\item 修订算法在每个章节重新编号问题;
\item 增加符号说明页和附录页(如果不需要,请在.cls文件对应处注释掉即可);
\item  增加参考文献按国标 gbt7714-2015要求,只核对了常用的图书、中英文期刊,会议格式,其余未常使用的未进行核对(如有问题请改回gbt7714-2005);
\item  修订多个子图Caption居中问题;
\item 依据专家意见调整成果与致谢部分间距,并增加目录中的点密度;
\item 按照图书馆最新要求(2020年12月份),去除目录中红色边框;
\item 增加页眉信息:中南大学博士论文与右侧的章节名保持一致,以及无需号章节名保持一致;
\item 增加中英文摘要至目录,并保持与章节名昨对其;
\item 参考文献完全依照国标 gbt7714-2005,修正了部分 Bug,提供了新的引用命令;
\item 按照最新版本要求,在声明扉页前后各增加一页空白页,保证装订单独成页;
\item 章节标题居中,并改成‘第1章’样式;
\item 目录中,将原章节标题换成‘第几章’样式,字体按要求加粗;
\item 中文摘要到目录结束用罗马数字编写页码,小五号Times New Roman,居中;
\item 增加插图索引和表格索引;
\item 所有的章节题目和中英文摘要均按要求修改字体和间距;
\end{itemize}

\subsection{主要研究工作}
\textbf{博士和硕士模板选择说明:}
\begin{itemize}
	\item 当前模板默认是博士,学术型。
	\item 如选择硕士模板,只需要将对应的content/info.tex文件中,选择$\setminus Doctorfalse$ \% 硕士学位论文,注释掉对应的博士模板就行。
	\item 学术型和专业型,盲审和正常版本,公开和涉密版本,均是同样操作;
	\item 其它模板,可以根据自己需要修改CSUthesis.cls文件。
\end{itemize}

(1) 提供图片插入示例。

(2) 提供表格插入示例。

(3) 提供公式插入示例。

(4) 提供参考文献插入示例。

\subsection{论文组织结构}

全文内容共六章,具体内容组织如下:

第一章为绪论。

第二章为图片插入示例。

第三章为表格插入示例。

第四章为公式插入示例。

第五章为参考文献插入示例。

第六章总结与展望,总结了本文的主要工作,展望了下一阶段的研究方向。

\clearpage

\section{图像布局}


\subsection{单图布局}



\textbf{单图布局如图\ref{F.csu_single}所示。}

\begin{figure}[hbt]
\centering
\includegraphics[width=0.5\textwidth]{csu.png}
\caption{单图布局示例}
\label{F.csu_single}
\end{figure}

\subsection{横排布局}

\textbf{横排布局如图\ref{F.csu_row}所示。}

\begin{figure}[!htb]
    \centering
    \begin{subfigure}[t]{0.24\linewidth}
    	\captionsetup{justification=centering} %子图caption居中
        \begin{minipage}[b]{1\linewidth}
        \includegraphics[width=1\linewidth]{csu.png}
         \caption{}
        \end{minipage}
    \end{subfigure}
    \begin{subfigure}[t]{0.24\linewidth}
    \captionsetup{justification=centering} %子图caption居中
        \begin{minipage}[b]{1\linewidth}
        \includegraphics[width=1\linewidth]{csu.png}
        \caption{}
        \end{minipage}
    \end{subfigure}
    \begin{subfigure}[t]{0.24\linewidth}
    	\captionsetup{justification=centering} %子图caption居中
        \begin{minipage}[b]{1\linewidth}
        \includegraphics[width=1\linewidth]{csu.png}
        \caption{}
        \end{minipage}
    \end{subfigure}
    \begin{subfigure}[t]{0.24\linewidth}
    	\captionsetup{justification=centering} %子图caption居中
        \begin{minipage}[b]{1\linewidth}
        \includegraphics[width=1\linewidth]{csu.png}
        \caption{}
        \end{minipage}
    \end{subfigure}
    \caption{横排布局示例}
    \label{F.csu_row}
\end{figure}



\subsection{竖排布局}
\textbf{竖排布局如图\ref{F.csu_col}所示。}

\begin{figure}[!htb]
    \centering
    \begin{subfigure}[t]{0.15\linewidth}
        \begin{minipage}[b]{1\linewidth}
        \includegraphics[width=1\linewidth]{csu.png}
        \caption{}
        \end{minipage}
    \end{subfigure}\\
    \begin{subfigure}[t]{0.15\linewidth}
        \begin{minipage}[b]{1\linewidth}
        \includegraphics[width=1\linewidth]{csu.png}
        \caption{}
        \end{minipage}
    \end{subfigure}
    \caption{竖排布局示例}
    \label{F.csu_col}
\end{figure}



\subsubsection{竖排多图横排布局}

\begin{figure}[!htb]
    \centering
    \begin{subfigure}[t]{0.13\linewidth}
    	\captionsetup{justification=centering} %子图caption居中
        \begin{minipage}[b]{1\linewidth}
        \includegraphics[width=1\linewidth]{csu.png} \vspace{-1ex} \vfill
        \includegraphics[width=1\linewidth]{csu.png}
         \caption{}
        \end{minipage}
    \end{subfigure}
    \begin{subfigure}[t]{0.13\linewidth}
    	\captionsetup{justification=centering} %子图caption居中
        \begin{minipage}[b]{1\linewidth}
        \includegraphics[width=1\linewidth]{csu.png} \vspace{-1ex} \vfill
        \includegraphics[width=1\linewidth]{csu.png}
        \caption{}
        \end{minipage}
    \end{subfigure}
    \caption{竖排多图横排布局}
    \label{F.csu_col_row}
\end{figure}

\textbf{竖排多图横排布局如图\ref{F.csu_col_row}所示。注意看(a)、(b)编号与图关系。}


\subsubsection{横排多图竖排布局}



\begin{figure}[!htb]
    \centering
    \begin{subfigure}[t]{0.3\linewidth}
    	\captionsetup{justification=centering} %子图caption居中
        \begin{minipage}[b]{1\linewidth}
        \includegraphics[width=0.45\linewidth]{csu.png}
        \includegraphics[width=0.45\linewidth]{csu.png}
        \caption{}
        \end{minipage}
    \end{subfigure}\\
    \begin{subfigure}[t]{0.3\linewidth}
    	\captionsetup{justification=centering} %子图caption居中
        \begin{minipage}[b]{1\linewidth}
        \includegraphics[width=0.45\linewidth]{csu.png}
        \includegraphics[width=0.45\linewidth]{csu.png}
        \caption{}
        \end{minipage}
    \end{subfigure}
    \caption{横排多图竖排布局,斜体emph \emph{A},A,斜体texit \textit{A}}
    \label{F.csu_row_col}
\end{figure}

\textbf{横排多图竖排布局如图\ref{F.csu_row_col}所示。注意看(a)、(b)编号与图关系。}

\subsection{本章小结}
本章示例图片布局。

\clearpage


\section{表格插入示例}

\begin{table}[htb]
  \centering
  \caption{表格为三线表斜体emph \emph{A},A,斜体texit \textit{A}}
  \label{T.example}
  \begin{tabular}{llllll}
  \toprule
    & \emph{A}A \textit{A}  & B  & C  & D  & E \\
  \midrule
1 	& 212 & 414 & 4 		& 23 & fgw	\\
2 	& 212 & 414 & v 		& 23 & fgw	\\
3 	& 212 & 414 & vfwe		& 23 & 嗯	\\
4 	& 212 & 414 & 4fwe		& 23 & 嗯	\\
5 	& af2 & 4vx & 4 		& 23 & fgw	\\
6 	& af2 & 4vx & 4 		& 23 & fgw	\\
7 	& 212 & 414 & 4 		& 23 & fgw	\\
\bottomrule

\end{tabular}
\end{table}

\textbf{表格如表\ref{T.example}所示,latex表格技巧很多,这里不再详细介绍。}



\clearpage

\section{算法示例}


\begin{algorithm}  
	\caption{Fourier-Mellin Based KCF}  
	\label{alg:A}  
	\hspace*{0.02in}{\bf Input:}
	Image $I$\\preprocessed kernelized template $T_\kappa$\\
	\hspace*{0.02in}{\bf Output:} 
	scale $\sigma$, angle $\theta$ relation between $I$ and $T$ 
	
	\begin{algorithmic}[1] 
		\STATE{fourier transform: $F=\mathcal{F}(I)$}
		\STATE {high pass filter: $F_h=\mathcal{H}(F)$\\$\mathcal{H}(x,y)=(1.0-cos(\pi x)cos(\pi y))(2.0-cos(\pi x)cos(\pi y))$}
		\STATE {log-polar transform: $F_{lp}=\mathcal{L}(F_h)$}
		\STATE {apply kernel function: $F_\kappa=\mathcal{K}(F_{lp})$}
		\STATE {phase correlation: $(\Delta x, \Delta y)=\mathcal{C}(F_\kappa, T_\kappa)$}
		\STATE {resolove scale and rotation:\\
			$\theta=\alpha \Delta x$, $\sigma=log(\Delta y)$\\
			where $\alpha$ is translation factor of pixel translation on fourier domain and polar angle on origin image
		}
	\end{algorithmic}  
\end{algorithm}


  \begin{algorithm}
	\caption{算法示例}
	\label{alg:SPSH01}
	\renewcommand{\algorithmicrequire}{\textbf{Input:}}
	\renewcommand{\algorithmicensure}{\textbf{Output:}}
	
	\begin{algorithmic}[1]
		\REQUIRE 相关输入。。。。
		
		\ENSURE 相关输出。。。
		
		\STATE 算法描述  % 只占用一个行号
		\FOR{$i \gets 1\cdots N$}
		\STATE 算法描述
		\FOR{\textbf{each} $j \gets 1\cdots K$}
		\STATE 算法描述
		\ENDFOR
		\ENDFOR
		\REPEAT
		\REPEAT
		\STATE 令$\tau\gets\tau+1$
		\UNTIL 内循环迭代终止条件
		\STATE 。。。
		\UNTIL 外循环迭代终止条件
	\end{algorithmic}
\end{algorithm}

\textbf{如算法\ref{alg:A}所示,latex算法技巧很多。按需调整,这里不再详细介绍。}

\clearpage

\section{公式、定理、证明插入示例}

\begin{flalign}
\text{P1: } &\min_{\eta,R_u>0,R_d>0}\big\{ T_{\text{latency}}(\eta,R_u,R_d)\big\}  \\
&\text{s.t. }   0 \leq \eta \leq 1 \label{P1C1} 
\end{flalign}

\textbf{公式插入示例如公式(\ref{E.example})所示。}

\begin{equation}
\gamma_{x}=
\left\{
  \begin{array}{lr}
  0, & {\rm if}~~\;|x| \leq \delta \\
  x, & {\rm otherwise}
  \end{array}
\right.
\label{E.example}
\end{equation}

\begin{flalign}
&\text{P1:} \max_{\bigg\{\substack{P_{m,i},P_{n,i}\\
		q_{m,i},q_{n,i}\\ \forall n,m,i}\bigg\}}\big[R_{\text{sum}}(P_{m,i},P_{n,i},q_{m,i},q_{n,i},\forall n,m,i)\big],  \\
& \text{s.t.}~~q_{m,i}\in(0,1), q_{n,i}\in(0,1), \forall n,m,i, \label{allocons1} \\
& \hspace{1.5em} ~0\leq \sum_{i=1}^Rq_{n,i}P_{n,i}\leq P_n^{sum}, \forall n, \label{powercons1}  \\ 
& \hspace{1.5em} ~0\leq \sum_{i=1}^Rq_{m,i}P_{m,i}\leq P_m^{sum}, \forall m, \label{powercons2} \\ 
& \hspace{1.5em} ~\sum_{i=1}^Rq_{n,i}\leq 1,\sum_{i=1}^Rq_{m,i}\leq 1,\forall m,n, \label{RBcons} \\ 
& \hspace{1.5em} ~\sum_{n=1}^Nq_{n,i}\leq 1,\sum_{m=1}^Mq_{m,i}\leq 1,\forall i, \label{RBcons2} \\
& \hspace{1.5em} ~C_{m,BS,i}(P_{m,i},q_{m,i})\geq\varepsilon_{m,i},\forall m, \label{capcons}
\end{flalign}


  %%%%%%%%%%%%%% 示例 %%%%%%%%%%%%%%%%
公式子编号示例:
\begin{subequations}\label{Eq.(3-4)}
	\renewcommand{\theequation}{\theparentequation-\alph{equation}} % 定义编号方式
	\begin{align}
		&\varphi_{n,t}\in\left\{0,1\right\}, \forall n\in\mathcal{N}, t\in\mathcal{N}_{\mathrm{t}}, \label{Eq.(3-4a)} \\
		&\varphi_{n,t}\in\left\{0,1\right\}, \forall n\in\mathcal{N}, t\in\mathcal{N}_{\mathrm{t}}, \label{Eq.(3-4b)} \\
		&\varphi_{n,t}\in\left\{0,1\right\}, \forall n\in\mathcal{N}, t\in\mathcal{N}_{\mathrm{t}}, \label{Eq.(3-4c)}
	\end{align}
\end{subequations}

其中,公式\ref{Eq.(3-4a)}表示。

\begin{equation}    
	H_{j}= \mathrm{Concat}(\mathrm{GAP}({F}_{j}),\mathrm{GMP}({F}_{j})),
\end{equation}
\begin{equation}    
	\tilde{H}_{j-1,j}= \mathrm{Concat}(H_{j-1},H_{j}),j=5,
\end{equation}
\begin{equation}    
	p_c^{(i)}=\mathrm{Softmax}(\boldsymbol{P}_\theta(\tilde{H}_{j-1,j})),
\end{equation}

  %%%%%%%%%% 示例 %%%%%%%%%%%5
定理和证明环境说明:如”定理”使用小四黑体,编号随章节变化重新编号(如定理4-1),定理内容使用小四宋体,且内容行距与正文一致;”证明”无需编号,且以黑色小方块结尾。

\begin{theorem}\label{theorem1}
	\setlength{\baselineskip}{20pt}         % 基准行间距
	\renewcommand{\baselinestretch}{1.0}   % 几倍行间距
	开始定理。。。
\end{theorem}

\begin{proof*} %% 证明不编号
	\setlength{\baselineskip}{20pt}         % 基准行间距
	\renewcommand{\baselinestretch}{1.0}   % 几倍行间距
	开始证明。。。
	$\hfill\blacksquare$ %% 以黑方块结尾
\end{proof*}

\clearpage

\section{参考文献插入示例}

LaTeX\cite{lamport1994latex}插入参考文献最方便的方式是使用bibliography\cite{pritchard1969statistical},大多数出版商的论文页面都会有导出bib格式参考文献的链接,建议使用Jabref管理参考文献,把每个文献的bib放入``thesis-references'',然后用bibkey即可插入参考文献。

中文文献\cite{zh-book-1},注意手动编辑bibkey为英文的即可。

可以将文献标注为右上角\citess{shiweisong2019},只需要在现有的cite后加“ss”即可。

英文会议\cite{Kraus2021Current}, \cite{WuYangLuEtAl2021}.

英文期刊\cite{LuoZengYuanEtAl2016}, \cite{Wu2022Boosting}.


\textbf{特别强调:}从Google下载的bib也不一定全是对的,如发现有信息缺失,请下载原文核对。比如已发表的期刊,要包保证年、卷、标。

\textbf{注意:}如发现替换后的参考文献没有更新,请删除主文件夹下xxx.bbl文件,重新编译即可。

\clearpage


\section{总结与展望}

\noindent{纯数字编号}
\begin{enumerate}
 \item XXXXXXXXXX
 \label{item1}
 \item XXXXXXXXXX
 \item XXXXXXXXXX
\end{enumerate}
罗马编号
\begin{enumerate}[label=(\roman*)]
 \item XXXXXXXXXX
 \label{item2}
 \item XXXXXXXXXX
 \item XXXXXXXXXX
\end{enumerate}
括号编号
\begin{enumerate}[label=(\arabic*)]
 \item XXXXXXXXXX
 \label{item3}
 \item XXXXXXXXXX
 \item XXXXXXXXXX
\end{enumerate}
半括号编号
\begin{enumerate}[label=\arabic*)]
 \item XXXXXXXXXX
 \label{item4}
 \item XXXXXXXXXX
 \item XXXXXXXXXX
\end{enumerate}
小字母编号
\begin{enumerate}[label=\alph*)]
 \item XXXXXXXXXX
 \label{item5}
 \item XXXXXXXXXX
 \item XXXXXXXXXX
\end{enumerate}

引用测试,正如\ref{item1}、\ref{item2}、\ref{item3}、\ref{item4}、\ref{item5}所示

\subsection{工作展望}
手动编号 %(不推荐,无法被交叉引用)
\par
本课题针对XX,鉴于XXX,对XX进行了提高,但是XXX,所以有如下XX:

(1)目前XX虽然XX,但是XX仍然XX,所以XX仍然是一个值得XX的问题。

(2)随着XX,XX具有XX的问题,仍值得进一步XX。

(3)本课题在XX有了XX,但是XX的XX还存在XX,所以XX。


\clearpage
